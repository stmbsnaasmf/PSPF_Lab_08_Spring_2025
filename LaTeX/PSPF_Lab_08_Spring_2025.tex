\documentclass[12pt]{article}
\usepackage{graphicx}			    % Use this package to include images %Path relative to the main .tex file 
\graphicspath{ {./Images/} }
\usepackage{amsmath}			    % A library of many standard math expressions
\usepackage{mathtools}              % For Aboxed{} (https://tex.stackexchange.com/questions/346577/boxed-and-align)
% \usepackage[margin=1in]{geometry} % Sets 1in margins. 
\usepackage{fancyhdr}			    % Creates headers and footers
\usepackage{enumerate}              % These two packages give custom labels to a list
\usepackage[shortlabels]{enumitem}
\usepackage{hyperref}               % https://www.overleaf.com/learn/latex/Hyperlinks
\usepackage{xcolor}
\usepackage[svgnames]{xcolor}
\usepackage{float}
\usepackage{cmupint}                % For upright integrals. https://tex.stackexchange.com/questions/503527/how-to-write-upright-integrals-with-automatic-sizing
\usepackage{tikz}
\usetikzlibrary{trees}
\usepackage{titling}
\usepackage{minted}                 % For code blocks
\usemintedstyle{monokai}            % For code blocks
\definecolor{bg}{HTML}{282828}      % For code blocks, from https://github.com/kevinsawicki/monokai
\usepackage{nameref}
% \usepackage{mathtools, tccomicsans}
% \usepackage{comicsans}
\usepackage[main,largesymbols]{tccomicsans} % https://www.reddit.com/r/LaTeX/comments/1l5no5d/comment/mwm64ze/
\renewcommand*\contentsname{Summary}
% \renewcommand{\contentsname}{\centering \normalfont\normalsize Contents}
\renewcommand{\contentsname}{\centering \bfseries\Large Contents}
% \renewcommand{\cftaftertoctitle}{\hfill}

\hypersetup
{
    colorlinks=true,
    linkcolor=blue,
    filecolor=magenta,      
    urlcolor=cyan,
    %pdftitle={Overleaf Example},
    pdfpagemode=FullScreen,
}

% \title{OOP Lab Manual 05}
% \author{STM}
% \date{December 2024}

\begin{document}

\begin{titlepage}
    \centering

    \vspace*{-8em}
    \includegraphics[width=0.5\textwidth]{Bismillah.png}%\\[2cm]
    \vspace*{5em}

    
    \vspace*{1cm}

     \includegraphics[width=0.5\textwidth]{GU Tech 1685x1330.png}\\[2cm]

    \MakeUppercase{\Huge \textbf{GU TECH}}\\[1.5ex]
    
    \vspace*{1cm}
    
    \Huge Problem Solving \& Programming Fundamentals \\[1.5ex]
    \LARGE Lab 08 \\[2cm]

    % {\Large STM} \\ [2cm]

    {\Large \today}\\[1cm]
    
\end{titlepage}

\newpage

% \vspace*{4cm}
% \begin{center}
%     \Huge \textbf{Outline}
% \end{center}

% \begin{itemize}
%    \item \nameref{Functions}
% \end{itemize}

\tableofcontents

\newpage


\addcontentsline{toc}{part}{Recursion}
\part*{\hypertarget{recursion}{Recursion}} \label{recursion}

\begin{center}
    In order to understand \hyperlink{recursion}{recursion}, you must understand \hyperlink{recursion}{recursion}.
\end{center}

\noindent \hyperlink{recursion}{Recursion} is a programming technique where a function calls itself within its own definition to solve a problem. \\

\noindent The following is a simple example of a function that calls itself to calculate the factorial of a number. \href{https://github.com/stmbsnaasmf/PSPF_Lab_08_Spring_2025/blob/main/factorial.c}{GitHub link}.

\begin{minted}[bgcolor=bg, framesep=2mm]{cpp}
#include <stdio.h>

int factorial(int n)
{
    if (1 == n || 0 == n)
    {
        return 1;
    }
    else
    {
        return (n * factorial(n - 1));
    }
}

int main(int argc, char** argv)
{
    int n;
    scanf("%d", &n);
    printf("%d\n", factorial(n));

    return 0;
}
\end{minted}

\newpage

\noindent Here is a visualization of how the function works for the input $5$:

\begin{minted}[bgcolor=bg, framesep=2mm]{cpp}

factorial(5)
5 * factorial(4)
5 * 4 * factorial(3)
5 * 4 * 3 * factorial(2)
5 * 4 * 3 * 2 * factorial(1)
5 * 4 * 3 * 2 * 1
5 * 4 * 3 * 2
5 * 4 * 6
5 * 24
120

\end{minted}

\vspace{1cm}

\addcontentsline{toc}{section}{Explanation of Recursion}
\section*{Explanation of \hyperlink{recursion}{Recursion}}

\noindent A recursive function must:

\begin{enumerate}
    \item Call itself recursively.
    \item Have a base case that makes the function stop recursing.
    \item Have a recursive case that makes the function move towards the base case.
\end{enumerate}

\noindent \textbf{Note:} It is possible for a recursive function to have more than one base case or recursive case. \\

\newpage

\begin{minted}[bgcolor=bg, framesep=2mm]{cpp}
#include <iostream>
using namespace std;

int csl_r(int n)
{
    if (1 == n)
    {
        return 1;
    }
    else
    {
        if (0 == n % 2)
        {
            return 1 + csl_r(n / 2);
        }
        else
        {
            return 1 + csl_r((n * 3) + 1);
        }
    }
}

int main(void)
{
    int n = 3;
    scanf("%d", &n);
    printf("%d\n", csl_r(n));

    return 0;
}
\end{minted}

\newpage

\noindent Here is a visualization of how the function works for the input $3$:

\begin{minted}[bgcolor=bg, framesep=2mm]{cpp}

csl(3)
1 + csl(10)
1 + 1 + csl(5)
1 + 1 + 1 + csl(16)
1 + 1 + 1 + 1 + csl(8)
1 + 1 + 1 + 1 + 1 + csl(4)
1 + 1 + 1 + 1 + 1 + 1 + csl(2)
1 + 1 + 1 + 1 + 1 + 1 + 1 + csl(1)
1 + 1 + 1 + 1 + 1 + 1 + 1 + 1
1 + 1 + 1 + 1 + 1 + 1 + 2
1 + 1 + 1 + 1 + 1 + 3
1 + 1 + 1 + 1 + 4
1 + 1 + 1 + 5
1 + 1 + 6
1 + 7
8

\end{minted}


\begin{minted}[bgcolor=bg, framesep=2mm]{cpp}

3 10 5 16 8 4 2 1

03, 1
10, 2
05, 3
16, 4
08, 5
04, 6
02, 7
01, 8

\end{minted}













\newpage
\addcontentsline{toc}{part}{Pls ignore me}
\part*{Pls ignore me} \label{pls ignore me}

Look at the \textbf{following \textit{integral}} : 

\begin{align*}
    \int x \, \text{d}x &= \frac{ a \, \textit{a} \, b \, \textit{b} \, v\, c \ o \ x^2}{2} + c \\
    \intertext{}
    \Aboxed{\frac{\text{d}}{\text{d}x} x^2 &= 2 x^{2 - 1} = 2x} \\
\end{align*}


\newpage
\addcontentsline{toc}{part}{Lab Tasks}
\part*{Lab Tasks}



\newpage
\addcontentsline{toc}{section}{Section}
\section*{Section}


\addcontentsline{toc}{subsection}{Subsection}
\subsection*{Subsection}

\addcontentsline{toc}{subsubsection}{Subsubsection}
\subsubsection*{Subsubsection}






















\end{document}
